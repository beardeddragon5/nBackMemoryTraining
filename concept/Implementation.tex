% !TEX root = ./Masterdatei.tex

\section{Entwurf und Implementierung}
Mit Abschluss der Konzeption wird nun der Anwendungsentwurf vorgestellt und über
die Implementierung berichtet.

Es wird die MVVM-Architektur verwendet. Die Applikation wird in 4 größere
Abschnitte unterteilt, die möglichst voneinander getrennt sind und nur über
entsprechende Abstraktionsschichten miteinander kommunizieren.
\begin{itemize}[itemsep=0pt]
\item Einstellungen
\item Statistiken
\item Spiel
\item Datenbank
\end{itemize}
In diesen Abschnitten wird über Viewmodels bzw. Indents kommuniziert.
ViewModels werden verwendet, um zwischen Fragments einer Activity zu
vermitteln, während Intents im Kontext der Anwendung zum harten Wechsel zwischen
Programmabläufen verwendet wird.

Die einzelnen Gui-Komponenten werden ausschließlich in XML-Format erstellt. Für
die Befüllung mit entsprechenden Daten wurde jedoch verzichtet ein
entsprechendes Templating Framework zu verwenden, um die Anwendung nicht unnötig
aufzublähen.

Für die Anbindung wurde Room benutzt. Dabei handelt es sich um eine
Abstraktionsschicht zu den Funktionen von SQLite. Mitunter wird damit auch
die Migration von verschiedenen Version der Datenbank behandelt.

Die Gui-Komponenten sind aus dem Material Design Componenten von Google
entnommen worden.

\subsection{Activities und Fragmente}
In diesem Abschnitt werden die entsprechenden Activitys der Anwendung und
die dazugehörigen Fragments genauer erläutert.

\subsubsection{MainActivity}
Die Main-Activity ist die einfachste Activity und dient nur als Hub, um zwischen
den Abschnitten zu wechseln.

\subsubsection{ConfigActivity}
Die \texttt{ConfigActivity} wird dafür verwendet, um das SettingsFragment zu
starten und die Toolbar Aktionen abzufangen. Die Aktionen und dazugehörigen
Buttons werden in den Resourcen als \texttt{"menu/configmenu.xml"} hinterlegt.

\paragraph{SettingsFragment}
Dieses Fragment erbt von \texttt{PreferenceFragmentCompat}. Das ist eine
Klasse der neuen \texttt{androidx} Pakets für die Kompatibilität mit älteren
Android Versionen. Sie unterscheiden sich in verschiedenen Teilen,
so lässt sich unter anderem der Datentyp der einzelenen Preferences nicht
mehr im XML definieren. Daher wird dies hier
programmatisch eingestellt. Um den Aufwand möglichst gering zu halten,
wird auch zunächst angegeben, dass die einzelnen Einstellungen persistent
sind. Das führt dazu, dass die Werte aus den SharedPreferences gelesen werden.
Programmatisch werden diese jedoch, nach dem Aufbau der Gui, als nicht
persistent gekennzeichnet, damit die Toolbar Aktionen nicht umgangen werden.

\subsubsection{AbstractGameActivity}
In der AbstractGameActivity wird die Grundfunktionalität  der Spielmodi
\texttt{Time} und \texttt{Expression} implementiert. Dabei werden 3 verschiedene
Fragmente benutzt, um die verschiedenen Spielzustände darzustellen.

\paragraph{GameFragment}
Dieses Fragment zeigt die aktuelle Expression an. Außerdem gibt sie Informationen
über die den Antwortversatz (n) und den aktuellen Stand, je nach Spielmodi.

\paragraph{NextFragement}
Dieses Fragment enthält nur einen Button um entsprechend einen Zug weiter zu
gehen. Es wird benötigt, um die ersten Aufgaben anzuzeigen, die noch keine
Antwort verlangen.

\paragraph{NumpadFragment}
Dieses Fragment zeigt ein Nummernfeld von 1-9 an. Die einzelnen Zahlenfelder
färben sich entsprechend Abb. \ref{ablauf} beim Drücken eines Feldes.

\paragraph{TimeGameActivity}
Benutzt einen CountDownTimer der die verbleibende Zeit im ViewModel mitspeichert.
Dadurch wird bei einem recreate die Zeit am entsprechenden Zeitpunkt fortgeführt.

\paragraph{ExpressionGameActivity}
Überprüft und zeigt den aktuellen Übungsstatus an.

\subsubsection{ScoreActivity}
Die Score Activity wird zum Abschluss einer Übung gestartet und zeigt
verschiedene Leistungsmerkmale der Übung an. Unter anderem holt sie auch
die Highscores aus der Datenbank.

\subsubsection{StatsActivity}
Die StatsActivity besteht aus einem Tabbed View Pager und beinhaltet nur die
Initalisierung des Adapters.

\paragraph{StatsListFragment}
Dieses Fragment enthält eine Liste mit beliebiger Anzahl an Elementen. Da das
Laden aller Elemente nach längerer Benutzung der App schwierig werden kann,
werden hierfür immer nur eine gewisse Anzahl an Items geladen und angezeigt.
Für diese Funktionaliät wird ein \texttt{PagedListAdapter} verwendet.

\paragraph{StatsGraphFragment}
Dieses Fragment enthält einen Graphen, der die Statistiken
über den entsprechend ausgewählten Zeitabschnitt visuell darstellt. Für die graphische
Darstellung wurde die Bibiliothek \texttt{MPAndroidChart} benutzt.

\subsection{Datenzugriffsschicht}
Als Datenzugriffschicht wird Room verwendet. Room ermöglicht es, mit wenig
Aufwand eine entsprechende Datenbank mit Schemas zu befüllen. Für diesen
Zweck wurden die 4 Klassen erstellt.

\paragraph{Stats}
Diese Klasse enthält das Datenbank-Modell und besteht nur aus einer
Ansammlung von Eigenschaften mit entsprechenden Annotations, die man aus
der Room Dokumentation entnehmen kann. Die Annotations geben die
Spaltennamen, Indexe und andere Metainformationen wieder.

\paragraph{StatsDao}
Ist ein interface mit verschiedenen Methoden, die jeweils mit einem
SQL-Query annotiert sind. Dabei können die Parameter der Methode im Query
verwendet werden.

\paragraph{StatsDatabase}
Diese Klasse erbt von \texttt{RoomDatabase} und enthält die Initalisierung
der Datenbank und eine vorgegebene abstrakte Methode die ein \texttt{StatsDao}
Objekt zurückgibt.

\paragraph{DateConverter}
Da Room mit Objekten der Klasse \texttt{java.util.Date} nicht arbeiten kann,
müssen sie entsprechend konvertiert werden. In diesem Fall zu \texttt{Long}
als Unix Timestamp. Um den entsprechenden Konverter bekannt zu machen, wird
an \texttt{StatsDatabase} eine Annotation mit der Klassenreferenz hinzugefügt.


